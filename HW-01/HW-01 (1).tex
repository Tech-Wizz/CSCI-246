\documentclass[11pt]{article}

%Don't change any thing before \begin{document}
%They are not useful for now, but later when you try to add figures
%these might be useful. In fact if you use sth fancy, you might need
%to add more packages, or macros.
\usepackage{amssymb,amsmath}
\usepackage{times,psfrag,epsf,epsfig,graphics,graphicx}
\usepackage{algorithm}
\usepackage{algorithmic}

\begin{document}
\date{August 28}

\title{CSCI 246: Assignment~1~(6 points)}

\author{Kruize Christensen}

\maketitle


\noindent
This assignment is due on {\bf Monday, Aug 31, 8:30pm}. You will need to
use Latex to generate a single pdf file and upload it under {\em Assignment 1}
on D2L. There will be a penalty for not using Latex (to finish the assignment).
This is {\bf not} a group-assignment, so you must finish the assignment by
yourself.
\newline
 
\section*{Problem 1.}

\noindent
Let $A=\{c,d,f,g\}$, $B=\{j,f\}$ and $C=\{g,d\}$. Answer the following questions:
\newline

\noindent
(1.1) Is $B\subseteq B$? Why?
\newline
\newline
Yes, because every element in B is also in B
\noindent
%{\bf Answer:}~~ 
\newline

\noindent
(1.2) Is $C\subseteq A$? Why?  
\newline 
\newline
No, because C is not equal to A
\noindent
%{\bf Answer:}~~ ............
\newline

\noindent
(1.3) Is $C\subset A$? Why?
\newline
\newline
Yes, because every element in C is in A but C is not equal to A
\noindent
%{\bf Answer:}~~ ............
\newline

\newpage

\noindent
\section*{Problem 2.}

Let $A=\{a,b\}$, $B=\{1,2\}$ and $C=\{2,3\}$.
Using set-roster notation, write each of the following sets. (Enter your answers as comma-separated lists of ordered pairs.)
\newline

\noindent
(2.1) $(B\cap C)\times A$
\newline
\newline
B\cap C = {2}
\newline
\{2\} x A = \{(2,a),(2,b)\}
\noindent
\newline
\newline
(2.2) $(A\times B)\cap (B\times C)$
\newline
\newline
A x B = \{ ( a , 1 ) , ( b , 1 ) , ( a , 2 ) , ( b , 2 )\}
\newline
B x C = { ( 1 , 2) , ( 2 , 2 ) , ( 1 , 3 ) , ( 2 , 3 )}
HI
\newline
Then $(A\times B)\cap (B\times C ) = \{\}$ no objects belong to both $(\times B)$ and $( B \times C)$ because A contains only letters while B \& C contain only numbers
\newline
\newpage

\section*{Problem 3.}

Let the universal set be $\mathbb{R}$, the set of all real numbers, and let
$A=\{x\in\mathbb{R}|-3\leq x\leq 0\}$, $B=\{x\in\mathbb{R}|-1<x<2\}$, 
$C=\{x\in\mathbb{R}|7<x\leq 9\}$. Find the following:
\newline

\noindent
(3.1) $A\cup B$
 = \{ -3 , -2 , -1 , 0 , 1 \}
\newline

\noindent
(3.2) $A\cap B$
 = \{ 0 \}
\newline

\noindent
(3.3) $A^c$
 = \{ 1 , 8 , 9 \}
\newline

\noindent
(3.4) $A\cap C$
 = \{ \} no objects belong to both A \& C
\newline

\newpage

\section*{Problem 4.}

Let $\mathbb{Z}$ be the set of all integers and let
$$A_0=\{n\in \mathbb{Z}| n=4k,~\mbox{for~some~integer~$k$}\}$$
$$A_1=\{n\in \mathbb{Z}| n=4k+1,~\mbox{for~some~integer~$k$}\}$$
$$A_2=\{n\in \mathbb{Z}| n=4k+2,~\mbox{for~some~integer~$k$}\}$$
$$A_3=\{n\in \mathbb{Z}| n=4k+3,~\mbox{for~some~integer~$k$}\}.$$

Is $\{A_0,A_1,A_2,A_3\}$ a partition of $\mathbb{Z}$? Explain your reason.
\newline
\newline
Yes because each object/number can be represented as 4k , 4k + 1 , 4k + 2 or 4k + 3 which is found in each each of the A’s and represented by the quotient-remainder theorem n = dq + r and 0 \leq r < d

%\noindent
%{\bf Answer:}~~ .......................
%\newline
\newpage

\section*{Problem 5.}

Let $V_i=\{x\in\mathbb{R}|-\frac{1}{i}\leq x \leq \frac{1}{i}\}=[-\frac{1}{i},\frac{1}{i}]$ for each positive integer $i$. Find each of the following.
\newline

\noindent
(5.1) $\bigcup\limits_{i=1}^{4}V_i$
\newline
\newline
[-1,1]\cup[-\frac{1}{2},\frac{1}{2}]\cup[-\frac{1}{3},\frac{1}{3}]\cup[-\frac{1}{4},\frac{1}{4}]
\noindent
%{\bf Answer:}~~ .................
\newline
\newline

\noindent
(5.2) Are $V_1,V_2,V_3,\cdots$ mutually disjoint? Why?
\newline
\newline
They are not mutually disjoint because the larger I becomes the lesser values are inside of it.
\noindent
%{\bf Answer:}~~ ....................
\newline
\newline

\end{document}

