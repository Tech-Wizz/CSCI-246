\documentclass[11pt]{article}

%Don't change any thing before \begin{document}
%They are not useful for now, but later when you try to add figures
%these might be useful. In fact if you use sth fancy, you might need
%to add more packages, or macros.
\usepackage{amssymb,amsmath}
\usepackage{times,psfrag,epsf,epsfig,graphics,graphicx}
\usepackage{algorithm}
\usepackage{algorithmic}
\usepackage{xcolor}

\begin{document}
\date{}

\title{CSCI 246: Assignment~7~(6 points)}

%\author{Your Name Here}

\maketitle


\noindent
This assignment is due on {\bf Friday, Nov 06, 9:30pm}. It is strongly
encouraged that you use Latex to generate a single pdf file and upload it
under {\em Assignment 7} on D2L. But there will NOT be a penalty for not
using Latex (to finish the assignment). This could be a group-assignment,
so you could form a group with $\leq 3$ students (mathematically, this means
you can also do it by yourself).
\newline
 
\section*{Problem 1.}

\noindent
(1.1) How many odd integers are there from 1,000 to 9,999?
\newline
\newline
4500 odd integers
\newline
\newline
\noindent
(1.2) How many integers from 1,000 to 9,999 have distinct digits?
\newline
\newline
$9 * 9 * 8 * 7 = 4536$ possible integers with distinct digits
\newline
\newline


\noindent
(1.3) How many odd integers from 1,000 to 9,999 have distinct digits?
\newline
\newline
$5 * 8 * 8 * 7 = 2240$ possible odd integers with distinct digits
\newline
\newline


\noindent
(1.4) What is the probability that a randomly chosen four-digit integer has distinct digits?
\newline
\newline
Distinct Digits - $\frac{4536}{9000} = 50.4 $ percent
\newline


\noindent
(1.5) What is the probability that a randomly chosen four-digit integer has distinct digits and is odd?
\newline
Distinct digits and odd - $\frac{2240}{9000} = 24.89$ percent
\newpage

\section*{Problem 2.}

\noindent
At company $C$, passwords must be from 4-6 symbols long and composed from the
26 uppercase letters of the Roman alphabet, the ten digits 0-9, and the 14
special symbols !, @, \#, \$, \%, $^{\wedge}$, \&, *, (, ), -, +, \{, and \}.
\newline

\noindent
(2.1) How many passwords are possible if repetition of symbols is allowed?
\newline
\newline
\newline
$50^4 + 50^5 + 50^6 = 15943750000$ possible passwords
\newline
\newline
\newline
\newline

\noindent
(2.2) How many passwords contain no repeated symbols?
\newline
\newline
\newline
$\frac{50!}{46!} + \frac{50!}{45!} + \frac{50!}{44!} = 11701082400$ possible passwords
\newline
\newline
\newline
\newline

\noindent
(2.3) How many passwords have at least one repeated symbols?
\newline
\newline
\newline
$15943750000 - 11701082400 = 4242667600$ possible passwords
\newline
\newline
\newline
\newline
\newpage


\section*{Problem 3.}

\noindent
(3.1) How many integers from 1 through 1000 are multiples of 2 or multiples of 9?
\newline
\newline
\newline
$1000 = 2 * 500$ (multiples of 2)
\newline
$999 = 9 * 111$ (multiples of 9)
\newline
$990 = 18 * 55$ (multiples of 18)
\newline
$= 500 + 111 - 55 = 556$ integers
\newline
\newline
\newline


\noindent
(3.2) Suppose an integer from 1 through 1000 is chosen at random. Use the result of part (3.1) to find the probability that the integer is a multiple of 2 or a multiple of 9. (Enter your probability as a percent.)
\newline
\newline
\newline
$\frac{556}{1000} = 55.6$ percent
\newline
\newline
\newline

\noindent
(3.3) How many integers from 1 through 1000 are neither multiples of 2 nor multiples of 9?
\newline
\newline
$1000 - 556 = 444$ integers
\newline
\newpage

\section*{Problem 4.}

A programmer Sam writes 500 lines of (correct) computer code in 17 days. His
boss claims that among one of the 17 days Sam must have written at least 30
lines of codes. Explain why his boss can claim that.
\newline
\newline
There are 500 lines of code and 17 days to code it. $\frac{500}{17} = 29.42$
\newline
Which means that Sam has written 29 lines each day.
\newline
$29 * 17 = 493$ lines of code which is missing 7 lines.
\newline
\newline
Therefore if he writes 1 extra line of code on 7 of the 17 days, he will complete all 500 lines. 
\newline
\newline
He could also write all 7 extra lines of code on one day which makes the number of lines he wrote on one day be 36 lines. 
\newline
\newline
This proves that the only way he could write 500 lines in 17 days is by having at least one day where he wrote 30 or more lines of code.
\newpage

\section*{Problem 5.}

\noindent
(5.1) How many {\color{red}17}-bit strings contain exactly {\color{red}eight} 1's?
\newline
\newline
$\frac{17!}{8!(17-8)} = \frac{145860}{6} = 24310$ 17-bit strings.
\newline
\newline
\newline


\noindent
(5.2) How many {\color{red}17}-bit strings contain at least {\color{red}fourteen} 1's?
\newline
\newline
$\frac{17!}{14!(3!)} = \frac{4080}{6} = 680$ 
\newline
$\frac{17!}{15! * 2!} = 136$
\newline
$\frac{17!}{16! * 1!} = 17$
\newline
$680 + 136 + 17 + 1 = 834$ 17-bit strings.
\newline


\noindent
(5.3) How many {\color{red}17}-bit strings contain at least {\color{red}one} 1?
\newline
\newline
$2^{17} = 131072$
\newline
Must subtract one because one of the outcomes has no ones.
\newline
$131072 - 1 = 131071$ 17-bit strings.
\newline

\noindent
(5.4) How many {\color{red}17}-bit strings contain at most {\color{red}one} 1?
\newline
\newline
Number of ways to select 0 ones: 1
\newline
Number of ways to select 1 one: 17
\newline
$1 + 17 = 18$ 17-bit strings.
\newline
\newline

\end{document}
