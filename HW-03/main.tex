\documentclass[11pt]{article}

%Don't change any thing before \begin{document}
%They are not useful for now, but later when you try to add figures
%these might be useful. In fact if you use sth fancy, you might need
%to add more packages, or macros.
\usepackage{amssymb,amsmath}
\usepackage{times,psfrag,epsf,epsfig,graphics,graphicx}
\usepackage{algorithm}
\usepackage{algorithmic}

\begin{document}
\date{October 1st, 2020}

\title{CSCI 246: Assignment~3~(6 points)}

\author{Zach Wadhams, Kruize Christensen, Kai Dockens}

\maketitle


\noindent
This assignment is due on {\bf Thursday, Oct 1, 8:30pm}. It is strongly
encouraged that you use Latex to generate a single pdf file and upload it
under {\em Assignment 3} on D2L. But there will NOT be a penalty for not
using Latex (to finish the assignment). This could be a group-assignment,
so you could form a group with $\leq 3$ students (mathematically, this means
you can also do it by yourself).
\newline
 
\section*{Problem 1.}

\noindent
Prove that if $a$ and $b$ are odd integers, then $a^2+b^2$ is even.
\newline
\newline
Let $a = 2x+1$ and $b = 2y+1$ where x and y are any two integers
\newline
\newline
$a^2 + b^2 = (2x+1)^2 + (2y+1)^2$
\newline
\newline
$= 4x^2 + 4x + 1 + 4y^2 + 4y + 1$
\newline
\newline
$= 4(x^2 + y^2 + x + y) + 2$
\newline
\newline
$= 2 (2(x^2 + y^2 + x + y) + 1)$
\newline
\newline
Because $2(x^2 + y^2 + x + y)$ is an integer, let it be $m$ 
\newline
\newline
$a^2 + b^2 = 2m$ where m is an integer. Therefore $(a^2 + b^2)$ must be an even integer
%\noindent
%\noindent
%{\bf Answer:}~~ .......................
%\newline
%\newline
%\newline
%\newline
\newpage


\section*{Problem 2.}

\noindent
Reorder the premises in the argument to show that the conclusion follows as
a valid consequence from the premises. It might be helpful to rewrite the
statements in if-then form and replace some of them by their contrapositives.
{\em You can assume the premise: ``The arguments in these examples are not arranged in regular order like the ones I am used to".}
\newline

{\bf A}. When I work a logic example without grumbling, you may be sure it is one I understand.
\newline
\newline
1) If I can’t understand a logic example, I grumble
\newline
\newline
{\bf B}. No easy examples make my head ache.
\newline
\newline
2) If I get a headache then the example is not easy
\newline
\newline
{\bf C}. I can't understand examples if the arguments are not arranged in regular order like the ones I am used to.
\newline
\newline
3) If an argument belongs to these examples, then the arguments are not arranged in regular order like the ones I am used to
\newline
\newline
4)  If the arguments are not arranged in regular order like the ones I am used to, then I can’t understand
\newline
\newline
{\bf D}. I never grumble at an example unless it gives me a headache
\newline
\newline
5) If I grumble at an example, then it gives me a headache.
\newline
\newline
\newline
\newline
The correct order is then 3,4,1,5,2
\newline
\newline
$\therefore$~~~ These examples are not easy.
\newline
\newline


%\noindent
%{\bf Answer:}~~ .........
%\newline
\newpage

\section*{Problem 3.}

Sam worked hard last night; in fact, he slept for only three hours. He claimed that he found the following theorem:

For every integer $n>3$, if $n$ is even then $n^2+1$ is prime.
\newline

\noindent
Find a counterexample for Sam's claim.
\newline
\newline
Let n = 8
\newline
{\bf Answer:}~~ $8^2 = 64 + 1 = 65$ which is not a prime number. This means Sam's claim in false.

%\noindent
%{\bf Answer:}~~ .........
%\newline

\newpage

\noindent
\section*{Problem 4.}

Prove the following statement directly by definition.
\newline

The difference of any two rational numbers is a rational number.
\newline
\newline
Let x and y be two rational numbers. By definition of rationals, x = a/b and, 
\newline
y = c/d for some integers a, b, c \& d with, b $\neq 0$ and d $\neq 0$.
\newline
\newline
x - y = (ad - bc)/bd
\newline
\newline
Numerator and Denominator are integers because products and differences of integers are always integers and bd $\neq 0$ by the zero product property.
\newline
\newline
Therefore x - y is a rational number because it is the quotient of two integers with a non zero denominator.

%\noindent
%{\bf Answer:}~~ ...........
%\newline
\newpage


\section*{Problem 5.}

If $n$ is an integer and $n>1$, then $n!$ is the product of $n$ and every other positive integer that is less than $n$, e.g., $5!=5\times 4\times 3\times 2\times 1$.
\newline

\noindent
(5.1) Write 6! in standard factored form.
\newline
\newline
6! = $2^4 * 3^2 * 5$
\newline

%\noindent
%{\bf Answer:}~~ .................
%\newline
%\newline

\noindent
(5.2) Write 20! in standard factored form.
\newline
\newline
20! = $2^{18} * 3^8 * 5^4 * 7^2 * 11 * 13 * 17 * 19$
\newline

%\noindent
%{\bf Answer:}~~ ....................
%\newline
%\newline

\noindent
(5.3) Without computing the value of $(20!)^2$, determine how many zeros are at the end of the number when it is written in decimal form. Explain the reason.
\newline
\newline
Write $(20!)^2$ in factored form $10^{8} * (2^{28} * 3^{16} * 7^4 * 11^2 * 13^2 * 17^2 * 19^2)$
\newline
\newline
10 does not divide into $2^{28} * 3^{16} * 7^4 * 11^2 * 13^2 * 17^2 * 19^2$ and that means that the numbers of zeroes at the end is 8

%\noindent
%{\bf Answer:}~~ ....................
%\newline
%\newline

\end{document}