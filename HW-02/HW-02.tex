\documentclass[11pt]{article}

%Don't change any thing before \begin{document}
%They are not useful for now, but later when you try to add figures
%these might be useful. In fact if you use sth fancy, you might need
%to add more packages, or macros.
\usepackage{amssymb,amsmath}
\usepackage{times,psfrag,epsf,epsfig,graphics,graphicx}
\usepackage{algorithm}
\usepackage{algorithmic}

\begin{document}
\date{September 14, 2020}

\title{CSCI 246: Assignment~2~(6 points)}

\author{Kruize Christensen}

\maketitle


\noindent
This assignment is due on {\bf Monday, Sep 14, 8:30pm}. It is strongly
encouraged that you use Latex to generate a single pdf file and upload it
under {\em Assignment 2} on D2L. But there will NOT be a penalty for not
using Latex (to finish the assignment). This could be a group-assignment,
so you could form a group with $\leq 3$ students (mathematically, this means
you can also do it by yourself).
\newline
 
\section*{Problem 1.}

\noindent
Find exact values for each of the following quantities without using a calculator. 
\newline

\noindent
(1.1) $\log_2(512)=$
\newline
\newline
$512/2=256/2=128/2=64/2=32/2=16/2=8/2=4/2=2$
\newline
$log_2(2^9)=9$
\newline
\noindent
{\bf Answer:}~~ 9
\newline

\noindent
(1.2) $\log_2(4096)=$
\newline
\newline
$4096/2=2048/2=1024/2=512/2=256/2=128/2=64/2=32/2=16/2=8/2=4/2=2$
\newline
$log_2(2^{12}))=12$
\newline
\noindent
{\bf Answer:}~~ 12
\newline

\noindent
(1.3) $\log_2(8192)=$
\newline
\newline
$8192/2=4096/2=2048/2=1024/2=512/2=256/2=128/2=64/2=32/2=16/2=8/2=4/2=2$
\newline
$log_2(2^{13})=13$
\newline
\noindent
{\bf Answer:}~~ 13


\noindent
(1.4) $\log_2(1024\cdot 4096)=$?  And $\log_2(1024)+\log_2(4096)=$?
\newline
\newline
$\log_2(2^{10}\cdot 2^{12})=22$
\newline
$10+12=22$
\newline
\noindent
{\bf Answer:}~~ 22 and 22
\newline

\noindent
(1.5) $\log_2(\frac{4096}{64})=$? And $\log_2(4096)-\log_2(64)=$?
\newline
\newline
$\frac{4096}{64}=64//\log_2(2^6)=6$
\newline
$(\log_2(4096)=12)-(\log_2(64)=6)=6$
\newline
\noindent
{\bf Answer:}~~ 6 and 6


\newpage


\section*{Problem 2.}

\noindent
Let $A=\{4,5,7\}$ and $B=\{y,z\}$. Let $p_1$ and $p_2$ be the {\em projections} of $A\times B$ onto the first and second coordinates (components). That is,
for each pair $(a,b)\in A\times B$, $p_1(a,b)=a$ and $p_2(a,b)=b$. 
\newline

Answer the following questions:
\newline

\noindent
(2.1) Find $p_1(5,y)$ and $p_1(4,z)$.
\newline
\newline
\noindent
{\bf Answer:}~~ 5 and 4
\newline

\noindent
(2.2) What is the range of $p_1$?
\newline
\newline
%\noindent
{\bf Answer:}~~ \{5,4\}
\newline

\noindent
(2.3) Find $p_2(5,y)$ and $p_2(4,z)$.
\newline
\newline
%
%\noindent
{\bf Answer:}~~ y and z
\newline

\noindent
(2.4) What is the range of $p_2$?
\newline
\newline
%
%\noindent
{\bf Answer:}~~\{y,z\} 
\newpage

\noindent
\section*{Problem 3.}

Write the negation, contrapositive, converse, and inverse for the following statement.
\newline

If $x$ is at most 25, then $x$ is smaller than 25 or $x$ is equal to 25.
\newline
$x \leq 25$, then $x<25$ or $x=25$
\newline

\noindent
(3.1) Negation
\newline
\newline
%
%\noindent
{\bf Answer:}~~ $x \leq 25$, $x \nless 25$ and $ x \neq 25$
\newline
\newline

\noindent
(3.2) Contrapositive
\newline
\newline
{\bf Answer:}~~ if $x \nless 25$ and $x \neq 25$ , then $x \nleq 25$
\newline
\newline

\noindent
(3.3) Converse
\newline
\newline
{\bf Answer:}~~If $x<25$ or $x=25$, then $x \leq 25$
\newline
\newline

\noindent
(3.4) Inverse
\newline
\newline
{\bf Answer:}~~If $x \nleq 25$, then $x \nless 25$ or $x \neq 25$
\newline
\newline
\newpage

\section*{Problem 4.}

Peter worked hard last night; in fact, he slept for only two hours. He claimed that he found the following theorem:

$\forall x\in\mathbb{Z}, \frac{x-1}{x}$ is not an integer.
\newline

\noindent
(4.1) Find a counterexample for Peter's claim.
\newline
\newline
{\bf Answer:}~~ When $x=1$, $\frac{1-1}{1}=\frac{0}{1}=0$ and zero is an integer and when $x=-1$,$\frac{(-1)-1}{-1}=\frac{-2}{-1}=2$ and two is also an integer
\newline
\newline

\noindent
(4.2) Can you add a hypothesis to have a conditional statement so that Peter's conclusion is true?
\newline
\newline
{\bf Answer:}~~$\forall x\in\mathbb{Z},x\neq \{-1,1\} , \frac{x-1}{x}$
\newline
This now makes the statement true because it excludes the two numbers making the original stamen false
\newline
\newline
\newpage

\section*{Problem 5.}

Let $X$ be the set of all students at MSU, and let $M(s)$ be ``$s$ is a math major," $C(s)$ be ``$s$ is a computer science major," and let $P(s)$ be ``$s$ is a physics major." Express the following statements using quantifiers, variables, and the predicates $M(s)$, $C(s)$, and $E(s)$.
\newline

\noindent
(5.1) Some computer science students are not physics majors.
\newline
\newline

%\noindent
{\bf Answer:}~~ $\exists C(s)\land P(s)$
\newline
\newline

\noindent
(5.2) No mathematics students are also physics majors.
\newline
\newline

%\noindent
{\bf Answer:}~~ $\nexists M(s) \land P(s)$
\newline
\newline

\noindent
(5.3) There is a computer science student who is both math and physics major.
(This person would eventually have 3 BS degrees!)
\newline
\newline

%\noindent
{\bf Answer:}~~ $\exists! C(s) \land (M(s) \land P(s))$
%\newline
%\newline

\end{document}

