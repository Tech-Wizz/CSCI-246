
\documentclass[11pt]{article}

%Don't change any thing before \begin{document}
%They are not useful for now, but later when you try to add figures
%these might be useful. In fact if you use sth fancy, you might need
%to add more packages, or macros.
\usepackage{amssymb,amsmath}
\usepackage{times,psfrag,epsf,epsfig,graphics,graphicx}
\usepackage{algorithm}
\usepackage{algorithmic}

\begin{document}
\date{}

\title{CSCI 246: Assignment~5~(6 points)}

%\author{Your Name Here}

\maketitle


\noindent
This assignment is due on {\bf Thursday, Oct 22, 8:30pm}. It is strongly
encouraged that you use Latex to generate a single pdf file and upload it
under {\em Assignment 5} on D2L. But there will NOT be a penalty for not
using Latex (to finish the assignment). This could be a group-assignment,
so you could form a group with $\leq 3$ students (mathematically, this means
you can also do it by yourself).
\newline
 
\section*{Problem 1.}

\noindent
You have two parents, four grandparents, eight great-grandparents, and so forth.
\newline

\noindent
(1.1) If all your ancestors were distinct, what would be the total number of your ancestors for the past 41 generations (counting your parents' generation as the first generation)?
\newline
\newline
%\noindent
%{\bf Answer:}~~ 
%\newline
\newline

\noindent
(1.2) Assuming that each generation represents 25 years, how long (in years) is 41 generation?
\newline
\newline
%\noindent
%{\bf Answer:}~~
%\newline
\newline

\noindent
(1.3) The total number of people who have ever lived is approximately 10
billion, which is $10^{10}$. Compare this fact to (1.1), what can you deduce
about the family tree rooted at you?
\newline
\newline
\newline


\section*{Problem 2.}

\noindent
A polygon is {\em convex} means that given any two points on or inside the polygon, the segment joining the points lie entirely in the polygon.
Use mathematical induction to prove that for every integer $n\geq 3$, the interior angles of any $n$-sided convex polygon add up to $180(n-2)$ degrees.
\newline
\newline
P(n): the sum of the angles of an n-sided polygon is 180(n-2) degrees.
\newline
Proof for P(3)
\newline
Let n = 3, 180(3-2)=180 degrees
\newline
A polygon with 3 sides is a triangle and the sum of the angles of a triangle is 180 degrees.
\newline
\newline
Proof for P(n) is true for n = k + 1
\newline
By induction, the sum of the angles in a k-sided polygon is 180(k-2) degrees. 
\newline
\newline
180(k-2) + 180 = 180(k - 2 + 1) = 180(k - 1) = 180 ((k - 1) - 2)
\newline
Making it true for n = k + 1
\newline
\newline

This proves the induction "for all integers $n \geq 3$, the sum of the angles in any n-sided convex polygon adds up to 180(n-2) degrees.


%\noindent
%{\bf Answer:}~~ .........
%\newline
\newpage

\section*{Problem 3.}

The given sequence is defined recursively. Using (directed) iteration to solve it, simplify your answer if possible.

$t_k=t_{k-1}+3k+1$, for each integer $k\geq 1$,

$t_0=0$.
\newline


What is $t_n$=?
\newline
\newline
$t_{1} = t_{0} + 3 + 1 = 4$
\newline
$t_{2} = t_{1} + 6 + 1 = 11$
\newline
$t_{3} = t_{2} + 9 + 1 = 21$
\newline
$t_{4} = t_{3} + 12 + 1 = 34$
\newline
\newline
Then let $t_{n} = an^2+bn+c$
\newline
1) $4 = a + b + c$ 
\newline
2)$11 = 4a + 2b +c$ 
\newline
3) $21 = 9a +3b +c$ 
\newline
Take 2-1: $3a + b = 7$
\newline
Take 3-1: $8a + 2b = 17$
\newline
$8a + 14 - 61 = 14$
\newline
\newline
$2a = 3$ which means $a = \frac{3}{2}$
\newline
$b = \frac{5}{2}$
\newline
Therefore the implicit formula is $t_{k} = \frac{3}{2}k^2+\frac{5}{2}k $ for all $k \geq 0$



%\noindent
%{\bf Proof:}~~ ....
%\newline

\newpage

\noindent
\section*{Problem 4.}

Let $T(n)$ be the running time (in \# of steps) of Peter's program, where $n$ is the input size $n$.
Solve the following recurrence relation and prove your claim by induction.
\newline

$T(n)=2T(n/2)+n^2$,

$T(1)=1$.
\newline

%\noindent
%{\bf Answer:}~~

$T(n) = aT(\frac{n}{b}) + nk$
\newline
By master method
\newline
Let a = 2, b = 2, and k = 2
\newline
$b \cdot k = 4 = a$
\newline
This is case 2 in the master method which is:
\newline
\newline
$T(n) = \theta (n^2log2)$
\newline
\newline
Proof by induction:
\newline
\newline
$T(1) \leq d1$ let $d1 \geq 1$
\newline
$T(1) = 1$, for $n = 1$ it is true
\newline
\newline
$T(m) = 2T(\frac{m}{2}) + m^2 \leq 2d1(\frac{m}{2}) \cdot 2log(\frac{m}{2}) + m^2$
\newline
$= m^2 \cdot logm - m^2(d1 \cdot log(2-1)) \leq m^2 \cdot log(m)$
\newline
If $n \leq m$ that implied that it is also true for m
\newline
\newline
Therefore by $T(n) \leq d1 \cdot n^2 \cdot log(n)$ for all n
%\newline
%
%\noindent
%{\bf Proof:}~~ .................
%\newline
%\newline

\newpage


\section*{Problem 5.}

Let $T(n)$ be the running time (in \# of steps) of Sam's program, where $n$ is the input size $n$.
Solve the following recurrence relation and prove your claim by induction.
\newline

$T(n)=4T(n/2)+n^2$,

$T(1)=1$.
\newline
\newline
\newline
$T(n) = aT(\frac{n}{b}) + nk$
\newline
By master method
\newline
Let a = 4, b = 2, and k = 2
\newline
$b \cdot k = 4 = a$
\newline
This is case 2 in the master method which is:
\newline
\newline
$T(n) = \theta (n^2log2)$
\newline
\newline
Proof by induction:
\newline
\newline
$T(1) \leq d1$ let $d1 \geq 1$
\newline
$T(1) = 1$, for $n = 1$ it is true
\newline
\newline
$T(m) = 4T(\frac{m}{2}) + m^2 \leq 4d1(\frac{m}{2}) \cdot 2log(\frac{m}{2}) + m^2$
\newline
$= m^2 \cdot logm - m^2(d1 \cdot log(2-1)) \leq m^2 \cdot log(m)$
\newline
If $n \leq m$ that implied that it is also true for m
\newline
\newline
Therefore by $T(n) \leq d1 \cdot n^2 \cdot log(n)$ for all n


%\noindent
%{\bf Answer:}~~ ...........
%\newline
%
%\noindent
%{\bf Proof:}~~ .................
%\newline
%\newline

\end{document}